\documentclass{article}

\begin{document}
\title{Module 3 Short Problem Set}
\author{Max Robinson}
\date{}
\maketitle


\section{Problem 1}
\noindent \textit{Question}: Qualitatively explain the impact of using stemming on each of the following: \\
(a) vocabulary size; \\
(b) total number of postings in an inverted file; \\
(c) average posting list length? \\
The format of a good answer would be something like: With stemming XXXX \{increases, decreases, doesn’t change\} by \{a lot, a little, at all, roughly zz\%\} because of YYYY. I want to see a statement about the effect, it's magnitude, and a clear rationale.





\section{Problem 2}
\noindent \textit{Question}: Express the numbers \{8, 14, and 513\} three ways: using a 12-bit binary representation, and the gamma and
delta codes. You must follow the method for computing gamma/delta described in the text and presented in the lecture
materials. I strongly recommend learning to do this by hand, but you may write (and provide) a short computer
program if you prefer – but do not use a program that you did not write yourself.


\section{Problem 3}
\noindent \textit{Question}: Below is a bit sequence for a gamma encoded gap list (as described in Chapter 5 of IIR and the lecture
materials). Decode the gap list and reconstruct the corresponding list of docids. Spaces are added for ease of reading --
the final part only has two bits. Hint: there are four docids.
1111 1100 0100 0111 1111 0010 0000 1111 1010 1011 1111 0100 00

\section{Problem 4}
\noindent \textit{Question}: True or False -- Any bit sequence (i.e., any combination of zeros and ones) can be interpreted as a valid
gamma encoded list of integers? Explain why this is true, or give an example showing that it is not.

\section{Problem 5}
\noindent \textit{Question}: Below is a bit sequence for a set of gaps encoding using Variable Byte encoding as described in Chapter 5 of
IIR. Decode the list of gaps and reconstruct the corresponding list of docids. Hint: there are three docids.
1100 0001 0000 0011 1011 0011 0000 0100 0001 1111 1000 0011

\end{document}