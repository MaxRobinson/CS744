\documentclass{article}

\begin{document}
\title{Module 2 Short Problem Set}
\author{Max Robinson}
\date{}
\maketitle


\section{Problem 1}
\textit{Question}: Character n-gram overlap is used for both automated spelling correction and personal name matching (i.e.,
deciding whether two names might be the same, a common database problem known as “record linkage”). Using a
character 3- gram representation, how many n-grams do “MISSISSIPPI” and “MISSISSIPI” have in common (the latter
is missing a 'P')? What is the Dice-coefficient score for these two strings using 3-grams? What is the Dice score using
2-grams instead? Which score is higher? Note: although there is nothing conceptually wrong in doing so, for this
problem, do not use “padded” n-grams (e.g., ‘\$’ or ‘\_’ symbols marking the beginning and end of the strings).
\newline 


3-Grams for ``MISSISSIPPI": MIS, ISS, SSI, SIS, ISS, SSI, SIP, IPP, PPI

3-Grams for ``MISSISSIPI": MIS, ISS, SSI, SIS, ISS, SSI, SIP, IPI

There are \textbf{7} 3-grams in common: MIS, ISS, SSI, SIS, ISS, SSI, SIP

Dice Co-efficient = $7/(9+8) = 7/17 = 0.412 $
\newline 

2-Grams for ``MISSISSIPPI": MI IS SS SI IS SS SI IP PP PI

2-Grams for ``MISSISSIPI": MI IS SS SI IS SS SI IP PI

Dice Co-efficient for 2-grams = $8/(10+9) = 8/19 = 0.421$
\newline

The score for the 2-grams is higher. 


\section{Problem 2}
\textit{Question}: Compute the edit distance (or Levenshtein distance) for these two pairs of strings: (a) ``EYESCREAM" and
``ICECREAM"; and (b) ``BROKENSTONE" and ``BOOKSTORES". Then report a sequence of transformations for that
cost that converts one string into the other. You should use unit costs for each operation: insertion, deletion, or
substitution; that is, each step has a cost of 1. Note, you do not need to write a program or produce any code for this
problem – these examples can be easily determined by pen and paper – you do not need to construct a table as the
example in the textbook.
\newline
\pagebreak

EYESCREAM to ICECREAM \\
Distance = 3 \\
delete pos 3 $\rightarrow$  EYECREAM \\
replace pos 0 with I $\rightarrow$ IYECREAM \\
replace pos 1 with C $\rightarrow$ ICECREAM \\


BROKENSTONE to BOOKSTORES \\
Distance = 5 \\
replace pos 1 with O $\rightarrow$ BOOKENSTONE  \\
delete pos  4 $\rightarrow$  BOOKNSTONE \\
delete pos  4 $\rightarrow$  BOOKSTONE \\
replace pos 7 with R $\rightarrow$  BOOKSTORE \\
insert pos 9 (at end) an S $\rightarrow$  BOOKSTORES \\


\section{Problem3}
\textit{Question}: Following the method described in the textbook (or lecture materials), what are the Soundex codes for the
strings: (a) "Jelinek" and (b) “Khudanpur”? Show your intermediate steps to produce the code.
\newline 

``Jelinek" \\ 
without AEIOUHWY: J0l0n0k \\
replace CGJKQSXZ: J0l0n02 (Note: we leave the first uppercase letter)\\
replace L: J040n02 \\ 
replace MN: J040502 \\
Remove all 0's: J452 \\
Final ``Jelinek" $\rightarrow$ J452
\newline

``Khudanpur" \\
without AEIOUHWY: K00d0np0r \\
replace DT: K0030np0r \\
replace MN: K00305p0r \\
replace B, F, P, V: K0030510r \\
replace R: K00305106 \\
Remove all 0's: K3516 \\
Truncate to get 4 chars: K351 \\
Final ``Khudanpur" $\rightarrow$ K351 \\

\end{document}






















