\documentclass{article}
\usepackage[left=3cm, right=3cm, top=2cm, bottom=2cm]{geometry, xcolor}

\begin{document}
\title{Short Problem Set 9}
\author{Max Robinson}
\date{}
\maketitle

\section{Problem 1}
\noindent \textit{Question}:  Describe in your own words the process described in the course text to efficiently identify near duplicate documents in a large collection.
\newline

\noindent
\textit{Answer}:


\section{Problem 2}
\noindent \textit{Question}:  Give a short definition or explanation of the following concepts:
\begin{enumerate}
\item web spam
\item near duplicate
\item in-degree
\item robot exclusion protocol
\item priority queue (in the context of web crawling)
\end{enumerate}


\noindent
\textit{Answer}:


\section{Problem 3}
\noindent
\textit{Question}: For this problem work with the directed web graph shown below. In the graph there are six nodes: S, B, F, G, T, and I (for the websites Snapchat, Bing, Facebook, Google, Twitter, and Instagram). Use a teleport probability of 0.25. Assume no other pages or links exist beside those shown in the figure. \\

\noindent
(a) Provide (i.e., write) the six recurrence equations that indicate how to iteratively calculate the
PageRank score of each page at time t given scores from time t-1.\\

\noindent
\textit{Answers}: \\
$PR(S, t_i) = 0.042 + .75*0 = 0.042$\\
$PR(B, t_i) = 0.042 + .75*(PR(S,t_{i-1})/2 + PR(G,t_{i-1})/2 + PR(F, t_{i-1})/1)$ \\
$PR(F, t_i) = 0.042 + .75*(PR(T,t_{i-1})/2)$ \\
$PR(G, t_i) = 0.042 + .75*(PR(S,t_{i-1})/2 + PR(B,t_{i-1})/2)$ \\
$PR(T, t_i) = 0.042 + .75*(PR(B,t_{i-1})/2 + PR(G,t_{i-1})/2)$ \\
$PR(I, t_i) = 0.042 + .75*(PR(T,t_{i-1})/2)$ \\

\noindent
(b) Using the brute-force iterative method of calculation shown in the video lecture calculate two
iterations of PageRank scores for each page in the graph. Be sure to show scores at times t=0, t=1, and finally
at t=2. Report scores using three digits of precision (e.g., 0.247, not 0.2 or 0.24696485932). Show work and
do not merely provide a table of values. \\


At $t=0$ let $PR(X) \approx 1/6$ since we have 6 pages. 
\begin{table}[h!]
	\begin{tabular}{lllllll}
		      & S    & B    & F    & G    & T    & I    \\
		$t=0$ & .167 & .167 & .167 & .167 & .166 & .166  
	\end{tabular}
\end{table}

\noindent
At $t=1$: \\
$PR(S, 1) = 0.042 + .75*0 = 0.042$\\

\noindent
$PR(B, 1) = 0.042 + .75*(PR(S,t_{i-1})/2 + PR(G,t_{i-1})/2 + PR(F, t_{i-1})/1)$ \\
$PR(B, 1) = 0.042 + .75*(.167/2 + .167/2 + .167/1)$ \\
$PR(B, 1) = 0.042 + .75*(0.334)$ \\
$PR(B, 1) = 0.293$ \\

\noindent
$PR(F, 1) = 0.042 + .75*(PR(T,t_{i-1})/2)$ \\
$PR(F, 1) = 0.042 + .75*(.166/2)$ \\
$PR(F, 1) = 0.042 + .75*(0.083)$ \\
$PR(F, 1) = 0.104$ \\

\noindent
$PR(G, 1) = 0.042 + .75*(PR(S,t_{i-1})/2 + PR(B,t_{i-1})/2)$ \\
$PR(G, 1) = 0.042 + .75*(0.167/2 + 0.167/2)$ \\
$PR(G, 1) = 0.042 + .75*(0.167)$ \\
$PR(G, 1) = 0.167$ \\

\noindent
$PR(T, 1) = 0.042 + .75*(PR(B,t_{i-1})/2 + PR(G,t_{i-1})/2)$ \\
$PR(T, 1) = 0.042 + .75*(0.167/2 + 0.167/2)$ \\
$PR(T, 1) = 0.042 + .75*(0.167)$ \\
$PR(T, 1) = 0.167$ \\

\noindent
$PR(I, 1) = 0.042 + .75*(PR(T,t_{i-1})/2)$ \\
$PR(I, 1) = 0.042 + .75*(.167/2)$ \\
$PR(I, 1) = 0.042 + .75*(0.0835)$ \\
$PR(I, 1) = 0.104$ \\

\begin{table}[h!]
	\begin{tabular}{lllllll}
		      & S    & B     & F     & G     & T     & I     \\
		$t=0$ & .167 & .167  & .167  & .167  & .166  & .166  \\
		$t=1$ & .042 & 0.293 & 0.104 & 0.167 & 0.167 & 0.104    
	\end{tabular}
\end{table}

\noindent
At $t=2$: \\
$PR(S, 2) = 0.042 + .75*0 = 0.042$\\

\noindent
$PR(B, 2) = 0.042 + .75*(PR(S,t_{i-1})/2 + PR(G,t_{i-1})/2 + PR(F, t_{i-1})/1)$ \\
$PR(B, 2) = 0.042 + .75*(0.042/2 + 0.167/2 + 0.104/1)$ \\
$PR(B, 2) = 0.042 + .75*(0.2085)$ \\
$PR(B, 2) = 0.198$ \\

\noindent
$PR(F, 2) = 0.042 + .75*(PR(T,t_{i-1})/2)$ \\
$PR(F, 2) = 0.042 + .75*(0.167/2)$ \\
$PR(F, 2) = 0.042 + .75*(0.0835)$ \\
$PR(F, 2) = 0.105$ \\

\noindent
$PR(G, 2) = 0.042 + .75*(PR(S,t_{i-1})/2 + PR(B,t_{i-1})/2)$ \\
$PR(G, 2) = 0.042 + .75*(0.042/2 + 0.293/2)$ \\
$PR(G, 2) = 0.042 + .75*(0.1675)$ \\
$PR(G, 2) = 0.168$ \\

\noindent
$PR(T, 2) = 0.042 + .75*(PR(B,t_{i-1})/2 + PR(G,t_{i-1})/2)$ \\
$PR(T, 2) = 0.042 + .75*(0.293/2 + 0.167/2)$ \\
$PR(T, 2) = 0.042 + .75*(0.23)$ \\
$PR(T, 2) = 0.215$ \\

\noindent
$PR(I, 2) = 0.042 + .75*(PR(T,t_{i-1})/2)$ \\
$PR(I, 2) = 0.042 + .75*(0.167/2)$ \\
$PR(I, 2) = 0.042 + .75*(0.0835)$ \\
$PR(I, 2) = 0.105$ \\

\begin{table}[h!]
	\begin{tabular}{lllllll}
		& S     & B     & F     & G     & T     & I     \\
		$t=0$ & .167 & .167  & .167  & .167  & .166  & .166  \\
		$t=1$ & .042 & 0.293 & 0.104 & 0.167 & 0.167 & 0.104  \\
		$t=2$ & 0.042 &  0.198 &  0.105 & 0.168 & 0.215 & 0.105
	\end{tabular}
\end{table}

\noindent
(c) Which page (or pages) has/have the lowest PageRank score after two iterations? \\
S(Snapchat) has the lowest PageRank score after two iterations.\\


\noindent
(d) Which page (or pages) has/have the highest PageRank score after two iterations?\\
T(Twitter) has the highest PageRank score after two iterations. \\

\noindent
(e) For part e only, suppose that we set the teleport probability in the PageRank equation to zero (0)
and we let the PageRank algorithm run for many iterators (say 50 iterations). Explain succinctly what the
important consequence or consequences of this choice would be? Justify your reasoning.



\end{document}







